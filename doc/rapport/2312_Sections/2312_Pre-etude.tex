\cleardoublepage
\chapter{Pré-étude}

%%%%%%%%%%%%%%%%%%%%%%%%%%%%%%%%%%%%%%%%%%%%%%%%%%%%%%%%%%%%%%%%%%%%%%%%%%%%%%%%%%%%%%%%%%
\section{RFID}

Des badges RFID sont mis à disposition des élèves pendant toute la durée de leur formation au sein de l'ETML-ES. Ceux-ci seront utilisés dans ce projet afin d'éviter aux élèves la nécessité de multiples badges. \\

La technologie du badge a pu être identifiée en utilisant un smartphone (Samsung S23 Ultra) doté de l'application "NFC Tools" (Version 8.9) disponible sur le "Play Store". \\

La figure \ref{fig:screenshotnfctools} illustre le standard adopté par le badge, en mettant en évidence le fabricant ainsi que le modèle de la puce interne. Des informations techniques plus détaillées sont également disponibles sur le site web du fabricant. 

\begin{figure}[h]
	\centering
	\includegraphics[width=0.7\linewidth]{2312_Images/2312_Pre-etude/Screenshot_NFC_Tools}
	\caption{Type de tag obtenu grâce à "NFC Tools"}
	\label{fig:screenshotnfctools}
\end{figure}

Je me suis ensuite mis en quête d'un lecteur RFID adapté à cette technologie. J'ai pu voir 

\href{https://www.mikroe.com/rfid-click}{Module MIKROE utilisé précédemment} \\

%%%%%%%%%%%%%%%%%%%%%%%%%%%%%%%%%%%%%%%%%%%%%%%%%%%%%%%%%%%%%%%%%%%%%%%%%%%%%%%%%%%%%%%%%%
\section{Alimentation}
L'entiereté du circuit sera alimenté par 3V3 car le microcontrôleur et le module rfid utilisent tout deux du 3V3. 
Donc j'utilise un module monobloc d'alimentation 230V à 3V3. 

%%%%%%%%%%%%%%%%%%%%%%%%%%%%%%%%%%%%%%%%%%%%%%%%%%%%%%%%%%%%%%%%%%%%%%%%%%%%%%%%%%%%%%%%%%
\section{Ethernet}

%%%%%%%%%%%%%%%%%%%%%%%%%%%%%%%%%%%%%%%%%%%%%%%%%%%%%%%%%%%%%%%%%%%%%%%%%%%%%%%%%%%%%%%%%%
\section{Boitier}
Le boitier sera réalisé en impression 3D. Cette méthode offre l'avantage d'une plus grande flexibilité.
Pour des raisons de sécurité, le plastique utilisé devra être résistant à la chaleur et non sensible à l'humidité. C'est le cas du PLA qui ne peut donc pas être utilisé dans cette application ou il y a de la haute tension.

%%%%%%%%%%%%%%%%%%%%%%%%%%%%%%%%%%%%%%%%%%%%%%%%%%%%%%%%%%%%%%%%%%%%%%%%%%%%%%%%%%%%%%%%%%
\section{Relais}
J'ai du faire le choix d'un relais pour commuter le 230VAC. J'ai choisi un relais capable de supporter un courant suffisamment élevé pour supporter le courant maximal que peut fournir la prise murale. J'ai dû m'assurer que la tension de contrôle soit aussi assez basse pour pouvoir correspondre à l'alimentation de mon circuit.
J'ai fait le choix aussi d'un relai à verrouillage pour permettre de réduire la consommation de courant après la commutation en conservant ainsi son état.

Le relais nécessite d'avoir une diode en série de ses bobines de set et reset (voir datasheet).
De plus, un circuit de protection doit être mis en place. Et il faut des mosfet pour les piloter.

%%%%%%%%%%%%%%%%%%%%%%%%%%%%%%%%%%%%%%%%%%%%%%%%%%%%%%%%%%%%%%%%%%%%%%%%%%%%%%%%%%%%%%%%%%
\section{Base de donnée}
La base de donnée sera géré par un Raspberry Pi. Cette solution offre la flexibilité d'un ordinateur avec une consommation minime. La gestion sera réalisé en langage Python.
